\Conclusion
    В ходе работы были изучены и реализованы алгоритмы нахождения
    расстояния Левенштейна (не рекурсивный с заполнением матрицы,
    рекурсивный без заполнения матрицы, рекурсивный с заполнением матрицы)
    и Дамерау-Левенштейна (не рекурсивный с заполнением матрицы). 
    Выполнено сравнение перечисленных алгоритмов. 
    
    В ходе экспериментов по замеру времени работы было установлено, что не рекурсивный алгоритм нахождения расстояния Левенштейна
    на длинах строк превышающих 3 на 136 \% быстрее, чем алгоритм поиска
    расстояния Левенштейна рекурсивно с заполнением матрицы и на 42 \%,
    чем реализация алгоритм поиска расстояния Дамерау-Левенштейна. На строках
    длиной менее 3х символов рекурсивная реализиция выигрывает матричные, так
    как не выделяет в куче место под хранение матрицы.
    
    Из теоритического анализа максимальной затрачиваемой памяти каждым из алгоритмов 
    представленым в технологической части можно сделать вывод, что реализации
    с использованием матриц занимают намного больше памяти при обработке
    длинных строк, чем рекурсивная реализация, так при длине строк 1000
    символов, рекурсивный алгоритм теоритически использует в 95.5 раз меньше памяти, чем остальные.
