\chapter{ Технологический раздел}
\label{cha:technological}

    В данном разделе приведены средства реализации и листинги кода.

    \section{Требование к ПО}
    
    К программе предъявляется ряд требований:
    
    \begin{itemize}
    	\item на вход подаются размеры матрицы, а также её элементы;
    	\item на выходе — матрица, которая является результатом получения среднего геометрического столбцов входной матрицы.
    \end{itemize}
    
    \section{Средства реализации}
    Для реализации ПО я выбрал язык программирования Си \cite{C}. Данный выбор обусловлен высокой скоростью работы языка, а так же наличия инструментов для создания и эффективной работы с потоками.
    
    \section{Реализация алгоритмов}
    
    В листингах 3.1 - 3.2 приведена реализация расмотренных ранее алгоритмов получения среднего геометрического столбцов матрицы.
    В листинге 3.3 приведена реализация функции создания и распределения потоков.
    
    \begin{lstlisting}[label=some-code,caption=Функция получения среднего геометрического столбцов матрицы обычным способом, language=C]
    void base_geometric_mean(args_t *args) {
        for (int j = 0; j < M; j++) {
            int ind_tmp_val = 1;
            for (int i = 0; i < N; i++) {
                args->res[ind_tmp_val][j] *= args->m1[i][j];
                if (i % 10 == 0 && i != 0) {
                    double n = N;
                    args->res[ind_tmp_val][j] = pow(args->res[ind_tmp_val][j], static_cast<double>(1) / n);
                    ind_tmp_val++;
                }
            }
            for (int ind = 1; ind < N / 10 + 2; ind++)
                args->res[0][j] *= args->res[ind][j];
        }
    }
    \end{lstlisting}
    
    \begin{lstlisting}[label=some-code,caption=Функция получения среднего геометрического столбцов матрицы параллельно.,language=C]
    void *parallel_geometric_mean_by_columns(void *args) {
        pthread_args_t *all_data = (pthread_args_t *)args;
    
        int col_start = all_data->thread_id * (all_data->matrix_size / all_data->cnt_threads);
        int col_end = (all_data->thread_id + 1) * (all_data->matrix_size / all_data->cnt_threads);
    
        for (int j = col_start; j < col_end; j++) {
            int ind_tmp_val = 1;
            for (int i = 0; i < N; i++) {
                all_data->matrix_container->res[ind_tmp_val][j] *= all_data->matrix_container->m1[i][j];
                if (i % 10 == 0 && i != 0) {
                    double n = N;
                    all_data->matrix_container->res[ind_tmp_val][j] =
                            pow(all_data->matrix_container->res[ind_tmp_val][j], static_cast<double>(1) / n);
                    ind_tmp_val++;
                }
            }
            for (int ind = 1; ind < N / 10 + 2; ind++)
                all_data->matrix_container->res[0][j] *= all_data->matrix_container->res[ind][j];
        }
        return NULL;
    }
    
    \end{lstlisting}
    
    \begin{lstlisting}[label=some-code,caption=Функция создания потоков,language=C]
    int start_threading(args_t *args, const int cnt_threads, const int type) {
        pthread_t *threads = (pthread_t *)malloc(cnt_threads * sizeof(pthread_t));
    
        if (!threads) {
            fprintf(stderr, "Ошибка при выделении памяти. Файл: %s\nСтрока: %d\n", __FILE__, __LINE__);
            return ALLOCATE_ERROR;
        }
    
        pthread_args_t *args_array = (pthread_args_t *)malloc(sizeof(pthread_args_t) * cnt_threads);
    
        if (!args_array) {
            free(threads);
            fprintf(stderr, "Ошибка при выделении памяти: %d\nФайл: %s\n", __LINE__, __FILE__);
            return ALLOCATE_ERROR;
        }
    
        for (int i = 0; i < cnt_threads; i++) {
            args_array[i].matrix_container = args;
            args_array[i].thread_id = i;
            args_array[i].matrix_size = N;
            args_array[i].cnt_threads = cnt_threads;
    
            switch (type) {
                case 1:
                    pthread_create(&threads[i], NULL, parallel_geometric_mean_by_columns, &args_array[i]);
                    break;
            }
        }
    
        for (int i = 0; i < cnt_threads; i++) {
            pthread_join(threads[i], NULL);
        }
    
        free(args_array);
        free(threads);
    
        return OK;
    }
    
    \end{lstlisting}
    
    \section{Тестовые данные}
    
    В таблице~\ref{tabular:test_rec} приведены тесты для функций, реализующих параллельное и обычное умножение матриц. Все тесты пройдены успешно.
    
    \begin{table}[h!]
    	\begin{center}
    		\begin{tabular}{c@{\hspace{7mm}}c@{\hspace{7mm}}c@{\hspace{7mm}}c@{\hspace{7mm}}c@{\hspace{7mm}}c@{\hspace{7mm}}}
    			\hline
    			Матрица & Ожидаемый результат \\ \hline
    			\vspace{4mm}
    			$\begin{pmatrix}
    			1 & 2 & 3 & 4\\
    			4 & 5 & 6 & 5\\
    			7 & 8 & 9 & 6
    			\end{pmatrix}$ &
    			$\begin{pmatrix}
    			3.036589 & 4.308869 & 5.451362 & 4.932424
    			\end{pmatrix}$ \\
    			\vspace{2mm}
    			\vspace{2mm}
    			$\begin{pmatrix}
    			3 & 4 & 2\\
    			27 & 16 & 8
    			\end{pmatrix}$ &
    			$\begin{pmatrix}
    			9 & 8 & 4
    			\end{pmatrix}$ \\
    			\vspace{2mm}
    			\vspace{2mm}
    			$\begin{pmatrix}
    			4
    			\end{pmatrix}$ &
    			$\begin{pmatrix}
    			4
    			\end{pmatrix}$ \\
    			\vspace{2mm}
    			\vspace{2mm}
    			$\begin{pmatrix}
    			0 & 0 & 0\\
    			1 & 2 & 3\\
    			1 & 2 & 3\\
    			1 & 2 & 3
    			\end{pmatrix}$ &
    			$\begin{pmatrix}
    			0 & 0 & 0 & 0
    			\end{pmatrix}$\\
    			\vspace{2mm}
    			\vspace{2mm}
    		\end{tabular}
    	\end{center}
    	\caption{\label{tabular:test_rec} Тестирование функций}
    \end{table}
    
    \section{Вывод}
    
    В данном разделе были разработаны исходные коды алгоритмов: обычный способ умножения матриц и два способа параллельного перемножения матриц.

\newpage