\chapter{Экспериментальный раздел}
\label{cha:research}
    
    \section{Технические характеристики}
    
    Ниже приведены технические характеристики устройства, на котором было проведено тестирование ПО:
    
    \begin{itemize}
    	\item Операционная система: Debian \cite{debian} Linux \cite{linux} 11 <<bullseye>> 64-bit.
    	\item Оперативная память: 12 GB.
    	\item Процессор: Intel(R) Core(TM) i5-3550 CPU @ 3.30GHz
    \cite{i5}.
    
    \end{itemize}
    
    \section{Время выполнения алгоритмов}
    В таблице 4.1 приведено сравнение реализации параллельного получения среднего геометрического столбцов матрицы с разным количеством потоков при размере исходной матрицы 512. В таблице 4.2 приведено сравнение однопоточной реализации и многопоточных (на четырёх потоках).
    
    \begin{table} [h!]
    	\caption{Таблица времени выполнения параллельных алгоритмов, при размере матрицы 512 (в тиках)}
    	\begin{center}
    		\begin{tabular}{|c c|} 
    			\hline
    			Количество потоков & Паралльная реализация алгоритма \\
    			\hline
    			1 & 16 352 377\\
    			\hline
    			2 & 8 413 109\\
    			\hline
    			4 & 6 617 493\\
    			\hline
    			8 & 5 469 821 \\
    			\hline
    			16 & 5 144 998 \\
    			\hline
    			24 & 5 825 791 \\
    			\hline
    			32 & 7 378 193 \\
    			\hline
    		\end{tabular}
    	\end{center}
    \end{table}
    
    \begin{table} [h!]
    	\caption{Таблица времени выполнения простого и параллельных алгоритмов (на 4 потоках) перемножения матриц (в тиках)}
    	\begin{center}
    		\begin{tabular}{|c c c|} 
    			\hline
    			Размер матрицы & Обычный & Параллельный\\  
    			\hline
    			64 & 209 360 & 766 202 \\
    			\hline
    			128 & 775 438 & 1 057 966 \\
    			\hline
    			256 & 2 575 187 & 2 485 468 \\
    			\hline
    			512 & 10 710 964 & 4 797 693 \\
    			\hline
    			1024 & 60 078 099 & 17 591 850 \\
    			\hline
    		\end{tabular}
    	\end{center}
    \end{table}
    
    \section{Вывод}
    
    Наилучшее время параллельные алгоритмы показали на 16 потоках, что соответствует количеству логических ядер компьютера, на котором проводилось тестирование. На матрицах размером 512 на 512, параллельные алгоритмы улучшают время обычной (однопоточной) реализации перемножения матриц примерно в 2.5 раза. При количестве потоков, большее чем 16, параллельная реализация замедляет выполнение (в сравнении с 16 потоками).
    

\newpage